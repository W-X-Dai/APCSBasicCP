\section{莫隊算法}
    \subsection{分塊}
    我們先討論一個線段樹的題目,求區間總和。
    今天換一個想法,不要把區間切成兩塊,而是切成大約
    $\sqrt{n}$塊。如此一來,建構的時間仍然是$O(n)$,
    查詢時,如果區塊完整包含直接回傳,沒有就一個一個數。
    這樣的複雜度是$O(\sqrt{n})$。

    你可能想說明明有更好的解,為什麼要講一個比較差的。
    因為,有時候我們不容易想到線段樹的合併方式,如果這時候
    可以使用離線操作的話,有機會使用莫隊算法。

    \subsection{概念}
    莫隊算法的概念就是區間分塊,不過,並不是像上面那一種,
    他是一個離線算法,換句話說就是可以調動查詢的順序。

    使用前,我們需要先分析題目,如果,對於已知的區間$[l,r]$,
    可以很快速的求得區間$[l-1,r]$以及$[l,r+1]$,則可以使用這個
    算法。

    首先,對所有查詢區間排序,排序時依照以下方法。

    \begin{itemize}
        \item 如果區間左端在不同區塊,則區間左端比較小的排前面。
        \item 否則區間右端較小的排前面。
    \end{itemize}

    這裡左右互換並不影響。接著,依照這樣的順序處理詢問,處理時,直接
    用$[l-1,r]$以及$[l,r+1]$推動區間到詢問的位置定紀錄答案。

    最後重新排序輸出即可。複雜度為$O(nt \sqrt{n})$,其中$t$是從已知的區間$[l,r]$,
    求得區間$[l-1,r]$以及$[l,r+1]$的複雜度。

    \example ZJ b417 區間眾數

    \textbf{題目敘述}

    給你$10^5$個數字以及$10^5$個詢問,對於每個詢問的區間,輸出
    那一個區間出現最多次的數字出現的次數,以及有多少種數字出現最多次。

    \textbf{想法}
    因為不容易使用線段樹,所以考慮使用莫隊。首先是如何使用已知的區間$[l,r]$,
    求得區間$[l-1,r]$以及$[l,r+1]$。我使用兩個std::map,第一個$f$
    存放每一個數字的出現的次數,第二個$ff$存放出現次數為$k$的數字有幾種。

    這樣可以在$O(\log n)$內完成區間移動,所以總時間複雜度為$O(n \log n \sqrt{n})$。
    當然如果用離散化可以把$\log$去掉。

\begin{lstlisting}[caption=區間眾數題解]
#define int long long
const int N=1000010;
using pii=pair<int,int>;

struct Query{
    int l,r,id;
    pii ans;
};

int n,m,k;
int a[N],f[N];
Query qq[N];

void solve(){
    cin>>n>>m;
    k=sqrt(n);
    for(int i=1;i<=n;++i){
        cin>>a[i];
    }

    for(int i=0;i<m;++i){
        cin>>qq[i].l>>qq[i].r;
        qq[i].id=i;
        qq[i].ans={0,0};
    }

    sort(qq,qq+m,[](Query a,Query b){
        return a.l/k!=b.l/k ? a.l/k<b.l/k : a.r<b.r;
    });

    map<int,int> ff;
    f[a[1]]++;
    f[a[2]]++;
    ff[f[a[1]]]++;
    ff[f[a[2]]]++;
    for(int i=0,l=1,r=2;i<m;++i){
        if(qq[i].l==qq[i].r){
            qq[i].ans={1,1};
            continue;
        }

        while(l!=qq[i].l || r!=qq[i].r){
            if(l<qq[i].l){
                ff[f[a[l]]]--;
                if(ff[f[a[l]]]<=0) ff.erase(f[a[l]]);
                f[a[l]]--;
                ff[f[a[l]]]++;
                l++;
            }

            if(l>qq[i].l){
                l--;
                ff[f[a[l]]]--;
                if(ff[f[a[l]]]<=0) ff.erase(f[a[l]]);
                f[a[l]]++;
                ff[f[a[l]]]++;
            }

            if(r<qq[i].r){
                r++;
                ff[f[a[r]]]--;
                if(ff[f[a[r]]]<=0) ff.erase(f[a[r]]);
                f[a[r]]++;
                ff[f[a[r]]]++;
            }

            if(r>qq[i].r){
                ff[f[a[r]]]--;
                if(ff[f[a[r]]]<=0) ff.erase(f[a[r]]);
                f[a[r]]--;
                ff[f[a[r]]]++;
                r--;
            }
        }

        auto it=ff.end(); it--;
        qq[i].ans=(*it);
    }

    sort(qq,qq+m,[](Query a,Query b){
        return a.id<b.id;
    });

    for(int i=0;i<m;++i){
        cout<<qq[i].ans.first<<" "<<qq[i].ans.second<<"\n";
    }
}
\end{lstlisting}

    \subsection{範例與練習}

    \problem 洛谷 P1494 [國家集訓隊] 小 Z 的襪子

    \textbf{題目敘述}

    作為一個生活散漫的人,小 Z 每天早上都要耗費很久從一堆五顏六色的襪子中找出一雙來穿。終於有一天,小 Z 再也無法忍受這惱人的找襪子過程,於是他決定聽天由命...

    具體來說,小 Z 把這 $N$ 只襪子從 $1$ 到 $N$ 編號,然後從編號 $L$ 到 $R$ ($L \leq R$) 的範圍內隨機抽取兩只襪子。儘管小 Z 並不在意兩只襪子是否能夠搭配成一雙,甚至不在意兩只襪子是否是左腳和右腳的,但他卻非常在意襪子的顏色,因為穿兩只不同顏色的襪子會很尷尬。

    你的任務是告訴小 Z 從範圍 $[L,R]$ 中抽取兩只襪子顏色相同的概率有多大。當然,小 Z 希望這個概率尽量高,所以他可能會提出多個範圍的詢問。

    然而如果 $L=R$,請特別處理這種情況,輸出 0/1。

    \textbf{輸入說明}

    輸入文件的第一行包含兩個正整數 $N$ 和 $M$。$N$ 表示襪子的總數量,$M$ 表示小 Z 的詢問數量。接下來一行包含 $N$ 個正整數 $C_i$,其中 $C_i$ 表示第 $i$ 只襪子的顏色,相同的顏色用相同的數字表示。接下來 $M$ 行,每行兩個正整數 $L$ 和 $R$,表示一個詢問範圍。

    $N$ 和 $M$ 不超過 50000,$1 \leq L < R \leq N$,$C_i \leq N$

    \textbf{輸出說明}

    輸出 $M$ 行,對於每個詢問,在一行中輸出一個最簡分數 $A/B$,表示從該詢問範圍 $[L,R]$ 中隨機抽取兩只襪子顏色相同的概率。如果該概率為 $0$,輸出 0/1。注意,輸出的分數必須是最簡分數。(請參考示例)

    \textbf{範例測試}

    \begin{tabular}{|m{7cm}|m{7cm}|}
        \hline
        範例輸入 1 & 範例輸出 1 \\
        \hline
        \verb|6 4| & \verb|2/5| \\
        \verb|1 2 3 3 3 2| & \verb|0/1| \\
        \verb|2 6| & \verb|1/1| \\
        \verb|1 3| & \verb|4/15| \\
        \verb|3 5| & \\
        \verb|1 6| & \\
        \hline
    \end{tabular}

    \problem 洛谷P1903 [國家集訓隊] 數顏色 / 維護隊列

    \textbf{題目敘述}

    墨墨購買了一套 $N$ 支彩色畫筆(其中有些顏色可能相同),擺成一排,你需要回答墨墨的提問。墨墨會向你發布如下指令:

    $Q\ L\ R$ 代表詢問你從第 $L$ 支畫筆到第 $R$ 支畫筆中共有幾種不同顏色的畫筆。
    
    $R\ P\ Col$ 把第 $P$ 支畫筆替換為顏色 $Col$。
    
    為了滿足墨墨的要求,你知道你需要幹什麼了嗎?

    \textbf{輸入說明}

    第 $1$ 行兩個整數 $N$,$M$,分別代表初始畫筆的數量以及墨墨會做的事情的個數。

    第 $2$ 行 $N$ 個整數,分別代表初始畫筆排中第 $i$ 支畫筆的顏色。
    
    第 $3$ 行到第 $2+M$ 行,每行分別代表墨墨會做的一件事情,格式見題干部分。

    \textbf{輸出說明}

    對於每一個 Query 的詢問,你需要在對應的行中給出一個數字,代表第 $L$ 支畫筆到第 $R$ 支畫筆中共有幾種不同顏色的畫筆。

    \textbf{範例測試}

    \begin{tabular}{|m{7cm}|m{7cm}|}
        \hline
        範例輸入 1 & 範例輸出 1 \\
        \hline
        \verb|6 5| & \verb|4| \\
        \verb|1 2 3 4 5 5| & \verb|4| \\
        \verb|Q 1 4| & \verb|3| \\
        \verb|Q 2 6| & \verb|4| \\
        \verb|R 1 2| & \\
        \verb|Q 1 4| & \\
        \verb|Q 2 6| & \\
        \hline
    \end{tabular}

    \begin{tip}
        想想看如果有修改應該要怎麼做。
    \end{tip}

    \problem 洛谷 P2464 [SDOI2008] 郁悶的小 J

    \textbf{題目敘述}

    小 J 是國家圖書館的一位圖書管理員,他的工作是管理一個巨大的書架。儘管他很能吃苦耐勞,但是由於這個書架十分巨大,所以他的工作效率總是很低,以致他面臨著被解雇的危險,這也正是他所鬱悶的。

    具體說來,書架由 $N$ 個書位組成,編號從 $1$ 到 $N$。每個書位放著一本書,每本書有一個特定的編碼。
    
    小 J 的工作有兩類:
    
    圖書館經常購置新書,而書架任意時刻都是滿的,所以只得將某位置的書拿掉並換成新購的書。
    
    小 J 需要回答顧客的查詢,顧客會詢問某一段連續的書位中某一特定編碼的書有多少本。
    
    例如,共 $N$ 個書位,開始時書位上的書編碼為 $A_1, A_2, \ldots , A_N$。
    
    一位顧客詢問書位 $1$ 到書位 $3$ 中編碼為“$K$”的書共多少本,得到的回答為:$X$。
    
    一位顧客詢問書位 $1$ 到書位 $3$ 中編碼為“$K$”的書共多少本,得到的回答為:$Y$。
    
    此時,圖書館購進一本編碼為“$P$”的書,並將它放到 $A$ 號書位。
    
    一位顧客詢問書位 $1$ 到書位 $3$ 中編碼為“$K$”的書共多少本,得到的回答為:$Z$。
    
    一位顧客詢問書位 $1$ 到書位 $3$ 中編碼為“$K$”的書共多少本,得到的回答為:$W$。
    
    ……
    
    你的任務是寫一個程式來回答每個顧客的查詢。

    \textbf{輸入說明}

    第一行兩個整數 $N, M$,表示一共 $N$ 個書位,$M$ 個操作。

    接下來一行共 $N$ 個整數數 $A_1, A_2, \ldots , A_N$,$A_i$ 表示開始時位置 $i$ 上的書的編碼。

    接下來 $M$ 行,每行表示一次操作,每行開頭一個字元。

    若字元為 C,表示圖書館購進新書,後接兩個整數 $A, P$($1 \le A \le N$),表示這本書被放在位置 $A$ 上,以及這本書的編碼為 $P$。

    若字元為 Q,表示一位顧客的查詢,後接三個整數 $A, B, K$($1 \le A \le B \le N$),表示查詢從第 $A$ 書位到第 $B$ 書位(包含 $A$ 和 $B$)中編碼為 $K$ 的書共多少本。

    $1 \le N, M \le {10}^5$,所有出現的書的編碼為不大於 $2^{31} - 1$ 的正數。

    \textbf{輸出說明}

    對每一位顧客的查詢,輸出一個整數,表示顧客所要查詢的結果。

    \textbf{範例測試}

    \begin{tabular}{|m{7cm}|m{7cm}|}
        \hline
        範例輸入 1 & 範例輸出 1 \\
        \hline
        \verb|5 5| & \verb|1| \\
        \verb|1 2 3 4 5| & \verb|1| \\
        \verb|Q 1 3 2| & \verb|0| \\
        \verb|Q 1 3 1| & \verb|2| \\
        \verb|C 2 1| & \\
        \verb|Q 1 3 2| & \\
        \verb|Q 1 3 1| & \\
        \hline
    \end{tabular}

    \problem 洛谷 P2709 小B的詢問

    \textbf{題目敘述}

    小B 有一個長為 $n$ 的整數序列 $a$,值域為 $[1,k]$。
    他一共有 $m$ 個詢問,每個詢問給定一個區間 $[l,r]$,求:

    $$\sum\limits_{i=1}^k c_i^2$$

    其中 $c_i$ 表示數字 $i$ 在 $[l,r]$ 中的出現次數。
    小B請你幫助他回答詢問。

    \textbf{輸入說明}

    第一行三個整數 $n,m,k$。

    第二行 $n$ 個整數,表示 小B 的序列。

    接下來的 $m$ 行,每行兩個整數 $l,r$。

    $100\%$ 的數據,$1\le n,m,k \le 5\times 10^4$

    \textbf{輸出說明}

    輸出 $m$ 行,每行一個整數,對應一個詢問的答案。

    \textbf{範例測試}

    \begin{tabular}{|m{7cm}|m{7cm}|}
        \hline
        範例輸入 1 & 範例輸出 1 \\
        \hline
        \verb|6 4 3| & \verb|6| \\
        \verb|1 3 2 1 1 3| & \verb|9| \\
        \verb|1 4| & \verb|5| \\
        \verb|2 6| & \verb|2| \\
        \verb|3 5| & \\
        \verb|5 6| & \\
        \hline
    \end{tabular}
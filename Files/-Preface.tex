\chapter*{前言}


林燈基金會贊助的蘭燈之星APCS競賽已經舉辦了數年,我也擔任了許多次的講師,
但每年的課程內容都要由講師重新設計,也沒有一套傳承好的教材可以使用。
這樣除了讓講師們每年都要花費大量時間準備課程外,
還會讓學生們在學習上遇到許多困難,因為每位講師的教學風格與教材內容都不盡相同。
此外,這樣也沒辦法起到傳承的效果,讓新講師可以快速上手。
因此,我決定編寫一套教材,讓未來的講師們可以輕鬆一些,
也期望未來的講師們能夠在此基礎上進行擴充與修改,
讓這套教材能夠更完善,並且能夠持續傳承下去。

本教材係基於李尚哲編寫的《APCS與競賽入門》一書進行改編,
主要是針對蘭燈之星APCS競賽的課程進行順序的調整以及擴充,
希望能夠加入一些在臺灣大學學到的內容。


演算法競賽的實力養成是一個漫長的過程,還望同學們可以持之以恆,
用對方法,開創出屬於你們的一片天地,在全國賽的戰場上向前衝。

\chapter*{上課規則}

本次課程的用意是讓同學們能夠學習到更深入的演算法,雖然說學習是自己的事,但為了維護課堂的風氣,還是訂定了以下的規則:

\begin{enumerate}
    \item 我們鼓勵你使用AI輔助學習,但請不要照抄AI的程式碼,比賽時禁止使用
    \item 任何的學習都得靠練習,因此我們會提供大量的練習題目給大家
    \item 如果遇到任何的問題之類的歡迎找我們討論,但\textbf{請不要拿AI的程式碼給我們debug}
    \item 關於優秀結業的標準如下:
    \begin{enumerate}
        \item 你行你上:在比賽中取得前三名並且解開80\%的比賽題
        \item 菜就多練:完成80\%的練習題目,並解開一半以上的比賽題
    \end{enumerate}
    \item 如果你有任何的問題或建議,歡迎隨時與我們聯繫!

\end{enumerate}
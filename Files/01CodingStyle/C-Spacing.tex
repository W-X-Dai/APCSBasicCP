\section{空格}
    \subsection{聲明}
    以下的建議僅供參考,空格的方式一直以來都是吵不完的議題,我將說明為何要這樣做。請依照你的個人需求選擇你的空格方式。(注意可讀性即可)

    \subsection{一般情況}
    因為會把程式碼的字體縮小,方便我閱讀更多行的程式碼,所以我會在運算符號間加上空格,這樣可以更容易懂。

    \begin{lstlisting}[caption=我的空格習慣(一般情況)]
void solve() {
    int a = 0, sum = 0;
    cout << sum << "\n";
}\end{lstlisting}

    \begin{lstlisting}[caption=不加上空格的版本]
void solve(){
    int a=0,sum=0;
    cout<<sum<<"\n";
}\end{lstlisting}


    \subsection{特殊情況}
    看到一般情況應該就會想到有特殊的情形。而為了可讀性,我在下列情形會加上空格。

    \begin{enumerate}
        \item 使用if而不想要換行時。
        \item 把兩個指令塞在同一行時。
        \item 運用指標時。
        \item 運算式太複雜時。
    \end{enumerate}
    
    以下程式是範例。
    
    \begin{lstlisting}[caption=特殊情況們]
void solve(){
    if(a<0) a=0;
    b++; c++;
    d->sum += e->sum;
    cout<<a + (b-c) - d->sum<<"\n";
}\end{lstlisting}

    不過盡可能避免這樣的特殊情況,因為這樣仍會讓程式碼變得難以閱讀。

    \begin{tip}
        請依照你習慣的方式,有一致性與可讀性的加上空格。你會發現本章節強調可讀性,因為我實在是看過好多沒有辦法閱讀的程式,我還必須手動排版,造成每個人時間的浪費。
    \end{tip}
\section{複雜度}
    複雜度的記法有三種,其中一種稱為O-notation,大多數情況都用大O表示法作為分析標準即可。大O表示法會著重於 $n$ 趨近時無窮大時函數的增長狀況。一般演算法分析為使其簡單易懂,有三個步驟省略紀錄,分別為:

    \begin{enumerate}
        \item 用緊實的上限(tight bound),簡單來說就是用盡可能小的複雜度(最有效率的方式)
        \item 忽略常數
        \item 忽略成長較慢的項
    \end{enumerate}

    以$f(n)=4n^3+n^2+3$為例,可將其表示為$O(n^3)$。

    複雜度分為時間與空間,一般來說空間是較為寬裕的,而時間限制較為緊湊。一般在多數狀況下時間限制多為1秒,以下是在這樣的時間限制下可以完成的資料規模參考。越後面是越不常見的。

    \begin{table}[h!]
        \centering
        \begin{tabular}{c|c|c|c|c|c|c|c}
            $O(1)$ & $O(\log(n))$ & $O(n)$ & $O(n \log(n))$ & $O(n^2)$ & $O(n^3)$ & $O(2^n)$ & $O(n!)$\\
            \hline
            $2^{63}-1$ & $2^{63}-1$ & $10^7$ & $10^6$ & $5 \times 10^3$ & $500$ & $20$ & $10$
        \end{tabular}
        \caption{複雜度與處理資料範圍對照}
        \label{tab:my_label}
    \end{table}

    關於複雜度,較為詳細的介紹在AP325有,大家可以參考。
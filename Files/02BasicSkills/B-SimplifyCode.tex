\section{巨集和程式碼簡化技巧}
    我個人其實並沒有大量地使用巨集,但是有許多人喜歡,因此我先列出一些較常用的以供參考。

    \begin{lstlisting}[caption=巨集們]
using namespace std;

// 縮短long long變數宣告以及常用常數
using ll=long long;
const ll MOD=1e9+7;
const ll INF=0x3f3f3f3f;
const ll LINF=0x3f3f3f3f3f3f3f3f;
const ll N=1000010;
const ll M=1010;
const long double PI=3.14159265358979;

// 巨集(競賽縮短程式碼技巧)
#define ALL(v) v.begin(),v.end()
#define siz(v) ((int)v.size())
#define F first
#define S second
#define EB emplace_back
#define PB pop_back
#define EF emplace_front
#define PF pop_front
#define EE emplace
#define rs resize
#define MP make_pair

template<typename T> using prior=priority_queue<T>;
template<typename T> using Prior=priority_queue<T,vector<T>,greater<T>>;
template<typename T> using Stack=stack<T,vector<T>>;
using pii=pair<int,int>;
using pll=pair<ll,ll>;\end{lstlisting}

    還有一些我完全沒有在使用的。

    \begin{lstlisting}[caption=更多巨集們]
using ld=long double;
template<typename T> using Stack=stack<T,vector<T>>;
template<typename T> using uset=unordered_set<T>;
template<typename T> using umap=unordered_map<T>;
template<typename T> using mset=multiset<T>;
template<typename T> using mmap=multimap<T>;
template<typename T> using umset=unordered_multiset<T>;
template<typename T> using ummap=unordered_multimap<T>;
template<typename T> using vit=vector<T>::iterator;
template<typename T> using sit=set<T>::iterator;
template<typename T> using mit=map<T>::iterator;
template<typename T> using usit=uset<T>::iterator;
template<typename T> using umit=umap<T>::iterator;
template<typename T> using msit=mset<T>::iterator;
template<typename T> using mmit=mmap<T>::iterator;
template<typename T> using umsit=umset<T>::iterator;
template<typename T> using ummit=ummap<T>::iterator;
using pdd=pair<ld,ld>;\end{lstlisting}

    \begin{tip}
        有人會說,這裡面有些不是巨集,沒有錯,你是對的,不過因為功能近似,所以我在這裡通稱他們巨集。
    \end{tip}
\section{樹直徑}
    \subsection{概念}
    樹直徑就是樹上的最長距離,例如前面的樹的示意圖,
    他的直徑就是5,從節點0一路走到節點8。

    如果要求樹直徑,我們有兩種方法,分別是兩次DFS以及動態規劃。
    若你還不知道動態規劃是什麼,你只要先知道可以透過把額外的資料
    存在陣列中,就可以得到樹直徑,這樣就可以了。

    \subsection{兩次DFS}
    兩次DFS可以用在邊權重不為負的情形。我們首先要隨便找一個點$n$。

    \begin{enumerate}
        \item 找到離n最遠的點u。
        \item 接著找到離u最遠的點v。
    \end{enumerate}

    找完以後$u \to v$就是這棵樹的其中一個直徑。我們可以在DFS的過程中,
    同時維護距離,如此就會在第二次DFS後直接完成計算。

\begin{lstlisting}[caption=兩次DFS]
struct pii{
    int mx,node;
};

pii dfs(int n,int p,int dis){
    pii ret={dis,n};
    for(auto u:g[n]) if(u.to!=p){
        pii tp=dfs(u.to,n,dis+u.dis);
        if(tp.mx>ret.mx){
            ret=tp;
        }
    }
    return ret;
}

int main(){
    // input
    pii a=dfs(1);
    pii b=dfs(a.node);

    cout<<b.mx<<"\n";
}
\end{lstlisting}

    \subsection{樹上DP}

    我們將會維護兩個陣列,分別是mx以及smx,
    裡面存放的是最長以及非嚴格次長,同樣以樹示意圖為例。
    將2設為根節點,同樣用DFS向下遞迴,我們看到節點1。
    他往下分別有節點5, 6, 7,其中節點6底下還有節點8。
    所以5, 6, 7底下的最長長度分別為0, 1, 0。

    於是我們可以將最長與次長的邊,透過節點1做為橋樑連接在一起,
    對所有節點都做過一次就可以找到直徑了。

\begin{lstlisting}[caption=樹直徑DP算法]
#define pii pair<int,int>
#define to first
#define dis second
const int INF=0x3f3f3f3f;

vector<pii> g[500010];
int mx[500010],smx[500010],ans=-INF;

void dfs(int x,int p){
    if(g[x].size()<=1) mx[x]=smx[x]=0;
    else mx[x]=smx[x]=-INF;
    for(auto i:g[x]) if(i.to!=p) {
        dfs(i.to,x);
        if(mx[x]<mx[i.to]+i.dis){
            smx[x]=mx[x];
            mx[x]=mx[i.to]+i.dis;
        }else if(smx[x]<mx[i.to]+i.dis){
            smx[x]=mx[i.to]+i.dis;
        }
    }
    ans=max(ans,mx[x]+smx[x]);
}

int main(){
    // input
    dfs(1,0);
    cout<<ans;
}
\end{lstlisting}

    \subsection{範例與練習}

    \problem 洛谷P3304 [SDOI2013]直徑

    \textbf{題目敘述}

    小Q最近學習了一些圖論知識。根據課本,有如下定義。樹:無迴路且連通的無向圖,
    每條邊都有正整數的權值來表示其長度。如果一棵樹有$N$個節點,可以證明其有且僅有$N-1$條邊。

    路徑:一棵樹上,任意兩個節點之間最多有一條簡單路徑。
    我們用 $dis(a,b)$表示點$a$和點$b$的路徑上各邊長度之和。
    稱$dis(a,b)$為$a,b$兩個節點間的距離。

    直徑:一棵樹上,最長的路徑為樹的直徑。樹的直徑可能不是唯一的。

    現在小Q想知道,對於給定的一棵樹,其直徑的長度是多少,
    以及有多少條邊滿足所有的直徑都經過該邊。

    \textbf{輸入說明}

    第一行包含一個整數$N$,表示節點數。 接下來$N-1$行,
    每行三個整數$a, b, c$ ,表示點$a$和點$b$之間有一條長度為$c$的無向邊。

    $2 \le N \le 200000$,所有點的編號都在$1 \cdots N$的範圍內,邊的權值$\le 10^9$。
    
    \textbf{輸出說明}

    共兩行。第一行一個整數,表示直徑的長度。第二行一個整數,表示被所有直徑經過的邊的數量。

    \textbf{範例測試}

    \begin{tabular}{|m{7cm}|m{7cm}|}
        \hline
        範例輸入 1 & 範例輸出 1 \\
        \hline
        \verb|6|        & \verb|1110| \\
        \verb|3 1 1000| & \verb|2| \\
        \verb|1 4 10|   &  \\
        \verb|4 2 100|  &  \\
        \verb|4 5 50|   &  \\
        \verb|4 6 100|  & \\
        \hline
    \end{tabular}

    \problem 洛谷P6722 「MCOI-01」村莊

    \textbf{題目敘述}

    今天,珂愛善良的0x3喵醬騎著一匹小馬來到了一個村莊。

    「嘿,這個村莊的佈局......」
    「好像之前我玩Ciste的地方啊qwq」

    0x3喵醬有一張地圖,地圖上有關於這個村莊的資訊。然後0x3喵醬要根據這張地圖來判斷Ciste是否有解。

    注:Ciste是《請問您今天要來點兔子嗎》中的一種藏寶圖遊戲。

    村莊被簡化為一個$n$個節點(編號為$1$到$n$)和$n-1$條邊構成的無向連通圖。

    0x3喵醬認為這個無向圖的資訊與滿足以下條件的新圖有關:

    新圖的節點集合與原圖相同
    在新圖中,節點$u$和節點$v$之間存在無向邊當且僅當在原圖中$dis(u,v) \ge k$($k$是給定的常數,$dis(u,v)$表示節點編號為$u$的節點到節點編號為$v$的節點的最短路徑長度)
    0x3喵醬還認為,如果這個"新圖"是二分圖,則Ciste有解;如果"新圖"不是二分圖,則Ciste無解。(如果您不知道二分圖,請參考提示)

    現在0x3喵醬想請您判斷一下這個Ciste是否有解。

    \textbf{輸入說明}

    第一行包含一個正整數$T$,表示有$T$組測試數據。
    對於每組測試數據,第一行包含兩個正整數$n$和$k$。接下來$n-1$行,每行包含三個正整數$x$、$y$和$v$,表示節點編號為$x$的節點到節點編號為$y$的節點有一條權重為$v$的無向邊。
    輸入數據保證合法。

    $n \le 10^5$,$T \le 10$,$v \le 1000$,$k \le 1000000$

    \textbf{輸出說明}

    對於每一組測試數據,輸出一行,如果Ciste有解則輸出"Yes",否則輸出"Baka Chino"。

    \begin{tip}
        二分圖又稱作二部圖,是圖論中的一種特殊模型。設$G=(V,E)$是一個無向圖,
        如果頂點$V$可分割為兩個互不相交的子集$(A,B)$,並且圖中的每條邊$(i,j)$所關聯的
        兩個頂點$i$和$j$分別屬於這兩個不同的頂點集$(i \in A, j \in B)$,
        則稱圖$G$為一個二分圖。
    \end{tip}

    \problem 實作一個演算法可以計算出次長樹直徑。
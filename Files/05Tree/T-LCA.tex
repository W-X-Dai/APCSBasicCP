\section{最近共同祖先}
    \subsection{概念}
    考慮一顆有根樹,我們希望找到任何兩個節點的最近同祖先(又稱為LCA)。
    我們以樹示意圖為例,將節點2定為根,我們希望找到節點5和8的最近共同祖先。
    那麼最近共同祖先便是1。

    我們可以很快的想到一個$O(n)$的方法找出任兩節點的最近共同祖先。
    首先,我們先跑過一次DFS預處理每個節點到根的距離,接著,
    如果需要查詢u和v的LCA,
    我們遵循幾個條件。

    \begin{itemize}
        \item 如果u離根的距離短,重複讓v與根的距離縮短,反之就讓u與根的距離縮短。
        \item 讓u,v與根的距離保持相等向上尋找。
        \item 如果找到同一個點就是正確的。
    \end{itemize}

\begin{lstlisting}[caption=$O(n)$ LCA算法]
int dis[N],par[N];
void dfs(int n,int p,int d){
    dis[n]=d;
    for(auto u:g[n]) if(u!=p){
        par[u]=n;
        dfs(u,n,d+1);
    }
}

int query(int u,int v){
    if(dis[u]>dis[v]){
        swap(u,v);
    }
    while(dis[v]>dis[u]){
        v=par[v];
    }

    while(u!=v){
        u=par[u];
        v=par[v];
    }

    return u;
}
\end{lstlisting}

    不過這樣的速度並不是很讓人滿意,有沒有辦法讓搜尋的速度提升呢,
    答案是可以的,只需要犧牲一點點預處理的時間就可以了。

    \subsection{倍增法}
    剛剛我們在找LCA的時候我們都是一個一個向上的,如果我們可以一次
    跳很多步,是否就會快上許多呢?沒有錯,我們可以要建構從節點u向上
    $1 \sim n$個節點是誰,在做二分搜,可是這樣預處理的時間將會是$O(n^2)$。
    太慢了。所以我們退而求其次,建構u向上$1,2,4,\cdots, 2^k, \cdots , 2^{\log_{2}{(n)}}$
    步的節點。這樣會花費$O(n\log{(n)})$的時間做預處理,也可以在$O(\log{(n)})$的時間內
    完成查詢的工作。

    \subsection{實作}

\begin{lstlisting}[caption=$O(\log{(n)})$查詢LCA算法]
vector<int> g[100010];
int lca[30][100010];
int lev[100010],root;

void init(int x=root,int p=root){
    lca[0][x]=p;
    lev[x]=lev[p]+1;
    for(auto i:g[x]) if(i != p) 
        init(i,x);
}

void build(int n){
    for(int i=1;i<=log2(n);++i)
        for(int j=1;j<=n;++j)
            lca[i][j]=lca[i-1][lca[i-1][j]];
}

int query(int a,int b){
    if(lev[a]<lev[b]) swap(a,b);
    for(int i=25;i>=0;--i)
        if(lev[lca[i][a]]>=lev[b])
            a=lca[i][a];
    if(a==b) return a;

    for(int i=25;i>=0;--i){
        if(lca[i][a]!=lca[i][b]){
            a=lca[i][a];
            b=lca[i][b];
        }
    }
    return lca[0][a];
}
\end{lstlisting}

    \subsection{範例與練習}

    \problem 洛谷P3379 【模板】最近公共祖先(LCA)

    \textbf{題目敘述}

    給定一棵有根多叉樹,求指定兩個節點的最近公共祖先。

    \textbf{輸入說明}

    第一行包含三個正整數 $N,M,S$,分別表示樹的節點個數、詢問的個數和樹的根節點序號。

    接下來 $N-1$ 行,每行包含兩個正整數 $x, y$,表示節點 $x$ 和節點 $y$ 之間有一條直接連接的邊(保證可以構成樹)。

    接下來 $M$ 行,每行包含兩個正整數 $a, b$,表示詢問節點 $a$ 和節點 $b$ 的最近公共祖先。

    $100\%$ 的數據,$1 \leq N,M \leq 500000$,$1 \leq x, y, a, b \leq N$,不保證 $a \neq b$
    
    \textbf{輸出說明}

    輸出包含 $M$ 行,每行包含一個正整數,依次為每一個詢問的結果。

    \textbf{範例測試}

    \begin{tabular}{|m{7cm}|m{7cm}|}
        \hline
        範例輸入 1 & 範例輸出 1 \\
        \hline
        \verb|5 5 4|  & \verb|4| \\
        \verb|3 1|  & \verb|4| \\
        \verb|2 4|  & \verb|1| \\
        \verb|5 1|  & \verb|4| \\
        \verb|1 4|  & \verb|4| \\
        \verb|2 4| & \\
        \verb|3 2|  & \\
        \verb|3 5|  & \\
        \verb|1 2|  & \\
        \verb|4 5|  & \\
        \hline
    \end{tabular}

    \problem CF 191 C Fools and Roads

    \textbf{題目敘述}

    他們說伯蘭德有兩個問題,愚蠢的人和道路。此外,伯蘭德有n個城市,由愚蠢的人居住,並由道路連接。伯蘭德的所有道路都是雙向的。
    由於伯蘭德有很多愚蠢的人,所以每對城市之間都有一條路徑(否則愚蠢的人會生氣)。此外,每對城市之間最多只有一條簡單路徑(否則愚蠢的人會迷路)。

    但這還不是伯蘭德的特點結束。在這個國家,愚蠢的人有時互相訪問,從而破壞了道路。愚蠢的人並不聰明,所以他們只使用簡單的路徑。

    簡單路徑是通過每個伯蘭德城市不超過一次的路徑。

    伯蘭德政府知道愚蠢的人使用的路徑。幫助政府計算每條道路上可以走的不同愚蠢的人數量。

    愚蠢的人的路徑將在在輸入中給出說明。

    \textbf{輸入說明}

    第一行包含一個整數$n(2 \le n \le 10^5)$——城市的數量。

    接下來的$n-1$行,每行包含兩個以空格分隔的整數$u_i$,$v_i$ $(1 \le u_i,v_i \le n,ui \ne vi)$,表示城市$u_i$和$v_i$之間有一條連接的道路。

    下一行包含整數$k(0 \le k \le 10^5)$——互相訪問的愚蠢人對數。

    接下來的$k$行包含兩個以空格分隔的數字。第$i$行$(i>0)$包含數字$a_i,b_i(1 \le a_i,b_i \le n)$。
    這意味著第$2i-1$個愚蠢的人住在城市$a_i$,並訪問住在城市$b_i$的第$2_i$個愚蠢的人。
    給定的數對描述了簡單的路徑,因為在每對城市之間只有一條簡單的路徑。
    
    \textbf{輸出說明}

    輸出$n-1$個整數,數字應以空格分隔。第$i$個數字應該等於第$i$條道路上可以走的愚蠢的人數量。
    道路從輸入中出現的順序開始編號,從1開始。

    \textbf{範例測試}

    \begin{tabular}{|m{7cm}|m{7cm}|}
        \hline
        範例輸入 1 & 範例輸出 1 \\
        \hline
        \verb|5|  & \verb|2 1 1 1| \\
        \verb|1 2|  & \\
        \verb|1 3|  & \\
        \verb|2 4|  & \\
        \verb|2 5|  & \\
        \verb|2| & \\
        \verb|1 4|  & \\
        \verb|3 5|  & \\
        \hline
    \end{tabular}
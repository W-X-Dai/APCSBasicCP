% %%%證明、結論、定義等等的環境,derived from CKMSG%%%
\newtheoremstyle{mystyle}% 自定義Style
  {6pt}{15pt}%       上下間距
  {}%               內文字體
  {}%               縮排
  {\bf}%            標頭字體
  {.}%              標頭後標點
  {1em}%            內文與標頭距離
  {}%               Theorem head spec (can be left empty, meaning 'normal')

% 改用粗體,預設 remark style 是斜體
\theoremstyle{mystyle}	% 定理環境Style
% \theoremstyle{definition}	% 定理環境Style
\newtheorem{theorem}{定理}[section]
\newtheorem{definition}[theorem]{Definition}
\newtheorem{example}[theorem]{Example}
\newtheorem{exercise}[theorem]{Exercise}
\newtheorem{corollary}[theorem]{Corollary}
\newtheorem{property}[theorem]{Property}
\newtheorem{proposition}[theorem]{Proposition}
\newtheorem{lemma}{Lemma}[section]
\newtheorem{problem}[theorem]{Problem}
\newtheorem{question}{Question}
%\newtheorem{tip}{Tip}[section] Need new name
\newtheorem*{remark}{Remark}
\newtheorem*{claim}{Claim}

    %% new commands %%
\newcommand*{\st}{\mbox{ such that }}
\newcommand*{\N}{\mathbb{N}}
\newcommand*{\Z}{\mathbb{Z}}
\newcommand*{\Q}{\mathbb{Q}}
\newcommand*{\R}{\mathbb{R}}
\newcommand*{\C}{\mathbb{C}}
% \newcommand*{\F}{\mathbb{F}}
% \DeclareMathOperator{\im}{Im}
% \DeclareMathOperator{\trace}{tr}
% \DeclareMathOperator{\vsspan}{span}
% \DeclareMathOperator{\image}{Im}
% \DeclareMathOperator{\rank}{rank}
% \DeclareMathOperator{\adj}{adj}
% \DeclareMathOperator{\ch}{ch}
% \DeclareMathOperator*{\ten}{\otimes}
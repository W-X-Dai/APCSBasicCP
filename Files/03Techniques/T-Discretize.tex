\section{離散化}
    \subsection{概念}
    離散化是排序的一個經典應用,可以將過大的值域範圍壓縮到$10^6$左右。
    首先,為什麼要壓縮值域,這是一個很好的問題,我們看以下範例。

    \example CSES 1619 Restaurant Customers

    \textbf{題目敘述}

    你有n個顧客在餐廳的到達和離開時間。請問在任何時間,餐廳中最多有多少顧客?

    \textbf{輸入說明}

    第一行輸入一個整數n:顧客數量。

    之後,有n行描述顧客。每行有兩個整數$a$和$b$:一個顧客的到達時間和離開時間。

    $1 \leq n \leq 2 \times 10^5$,$1 \leq a < b \leq 10^9$

    \textbf{輸出說明}

    輸出一個整數:最多顧客的數量。

    \textbf{想法}
    有一個很容易的解(我竟然還沒有寫差分,如果有空再補)就是標記所有進出時間,也就是在進入的時候+1,出去的時候-1,
    最後從頭到尾掃過陣列一次,並記錄最大值就可以了。

\begin{lstlisting}[caption=差分範例]
const int N=400010;

struct cus{
    int a,b;
};

int d[N];

int main(){
    // input ans pre-process
    
    for(int i=0;i<n;++i){
        d[cs[i].a]++;
        d[cs[i].b+1]--;
    }

    int psum=0,pmx=0;
    for(int i=1;i<N;++i){
        psum+=d[i];
        pmx=max(psum,pmx);
    }

    cout<<pmx<<"\n";
}
\end{lstlisting}

    但是我們看到a,b都有很大的範圍,這時候我們就可以用離散化將範圍縮小。
    其中,unique()函式可以將重複的東西刪除,resize()會重新分配記憶體空間
    給vector,讓大小符合刪除完之後的元素個數。

\begin{lstlisting}[caption=離散化範例]
const int N=400010;

struct cus{
    int a,b;
};

int d[N];

int main(){
    int n;
    cin>>n;

    vector<cus> cs(n);
    vector<int> dis;

    for(int i=0;i<n;++i){
        cin>>cs[i].a>>cs[i].b;
        dis.emplace_back(cs[i].a);
        dis.emplace_back(cs[i].b);
    }

    sort(dis.begin(),dis.end());
    dis.resize(unique(dis.begin(),dis.end())-dis.begin());

    for(int i=0;i<n;++i){
        cs[i].a=lower_bound(dis.begin(),dis.end(),cs[i].a)-dis.begin()+1;
        cs[i].b=lower_bound(dis.begin(),dis.end(),cs[i].b)-dis.begin()+1;
    }
}
\end{lstlisting}

    \textbf{另解}
    
    其實也可以透過\verb|map<int,int>|解決這個問題,有興趣的同學可以試試看。

    \subsection{注意事項}
    離散化只能用在壓縮值域的值並不是計算的重點時,因為如果計算上值很重要,那就會導致你
    更改到數值,進而吃WA。所以離散化只能用在,只需要考慮值的不同,或是大小順序即可的題目。

    另外就是像上面說的,大多時候都可以用其他方法替代(後面還有其他範例),所以也可以不用學(誤)。
    
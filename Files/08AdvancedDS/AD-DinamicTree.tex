\section{動態開點}
    \subsection{前言}
    其實前面的指標型線段樹就是了,不過還是讓我來帶你們了解一下原理。

    \subsection{概念}
    首先是為什麼要動態開點?當然是為了省記憶體。如果你有越多節點,
    你消耗的記憶體就越多。有時候會遇到數值範圍過大的情形,我們就可以做
    動態開點。那一般而言什麼是過大的範圍呢?只要超過$10^6$的級距就可以
    算是過大的範圍。

    有一個常見的問題是,為什麼不使用離散化(編到這邊我才發現我好像沒有編進第三章
    ,於是繼續趕工)。答案是因為,有時候我們需要確切的值,例如上一個單元的矩形覆蓋
    面積問題,如果你用離散化,那你的值就跑掉了,這樣就會算出錯誤的答案。

    \subsection{實作}
    其實之前那個就是了,所以我不重複貼上。需要注意的是,小心不要存到nullptr的節點
    ,因為會吃到RE,或SEG。還有空節點在計算的時候要小心,有些線段樹的空節點是有值的。
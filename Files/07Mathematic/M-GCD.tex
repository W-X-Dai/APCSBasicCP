\section{輾轉相除法}
    \subsection{概念}
    應該國中學過?就是反覆取餘數,然後交換位置。

\begin{lstlisting}[caption=GCD]
int gcd(int a,int b){
    return a==0 ? b : gcd(b,a%b);
}
\end{lstlisting}

    \subsection{Bézout's theorem}
    貝祖定理。對於所有$a, b \in \Z$,存在$x,y \in \Z$滿足$ax + by = \gcd(a,b)$。

    \subsection{擴展輾轉相除法}
    又稱為Extended Euclidean algorithm。可以找到
    $ax + by = \gcd(a,b)$。

    怎麼找呢?想像$a = 0,\ b = k$,則可以設$(x,y) = (0,1)$,
    如此一來就會滿足$ax + by = \gcd(a,b)$。

    對於$a \ne 0$的情況,因為$\gcd(a,\ b) = \gcd(b,\ a \% b)$
    
    所以
    
    \begin{align*}
        gcd(a, b) &= b \times x + (a \% b) \times y \\
        &= b \times x + (a - \lfloor \frac{a}{b}\rfloor \times b) \times y \\
        &= a \times (y) + b \times (x - \lfloor \frac{a}{b}\rfloor \times y) \\
        (x',\ y') &= (y,\ x - \lfloor\frac{a}{b}\rfloor\times y)
    \end{align*}

\begin{lstlisting}[caption=EXT GCD]
    // ax + by = gcd(a, b)
int extgcd(int a,int b,int &x,int &y){
    if(b==0){
        x=1;
        y=0;
        return y;
    }
    
    int xp,yp;
    int ret = extgcd(b, a%b, xp, yp); // xp * b + ap * (a%b) = gcd(a, b)
    
    x = yp;
    y = xp - yp * (a / b);
    return ret;
}
\end{lstlisting}

    求出一組解就可以求出所有解,因為如果$ax + by = \gcd(a,\ b)$,
    則$a(x + kb) + b(y - ka) = \gcd(a,\ b)$,其中
    $k \in \mathbb{Z}$。

    
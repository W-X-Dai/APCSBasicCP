\section{質數判斷}
    \subsection{簡單算法}
    就是從1跑到n-1,看看有沒有可以整除n的數。

    \subsection{小進步}

    \begin{theorem}
        $N = ab$ and  $a,b \in \N$
        $\implies min(a,b) \leq \sqrt{N}$
    \end{theorem}

    因此,我們可以跑$[1,\sqrt{N}]$就好。

\begin{lstlisting}[caption=$\sqrt{N}$ 質數判斷]
using ll=long long;
bool IsPrime(ll n){
    for(ll i=2;i*i<=n;++i){
        if(n%i==0){
            return false;
        }
    }
    return true;
}
\end{lstlisting}

    \subsection{埃氏篩法}
    國小應該就有教過?簡單說就是2是質數,
    所以把2的所有倍數刪掉,接著往後找,第一個沒被刪的數為3,
    所以一樣把所有3的倍數刪除,依此類推。

    實作上我們會使用bitset,如果要省記憶體。使用bool陣列,
    如果要快。(聽起來矛盾是吧,其實沒有,bitset快的只有位元運算)。

    複雜度為$\frac{N}{2} + \frac{N}{3} + \frac{N}{5} + \frac{N}{7} + \cdots$,
    其中N是篩的範圍。經過通靈(我不會證明的簡稱)我們得知其複雜度為$O(n \log{\log{n}})$

    關鍵字 Sum of reciprocals(倒數) of primes。

\begin{lstlisting}[caption=埃氏篩法]
using ll=long long;
const int N=1e7+10
bitset<N> isp;

void eratosthenes(){
    isp.set();
    isp[0]=isp[1]=false;
    
    for(ll i=2; i*i<=N; i++){
        if(isp[i]){
            for(ll j=i*i;j<=N;j+=i){
                isp[j]=false;
            }
        }     
    }
}
\end{lstlisting}

    \subsection{Millar-Rabin}

    神奇的東東,貼個程式碼就好,我也不知道原理。但複雜度為$O(\log^{3}{(n)})$

\begin{lstlisting}[caption=Millar-Rabin 質數判斷]
// n < 4,759,123,142        {3 : 2, 7, 61}
// n < 1,122,004,669,633    {4 : 2, 13, 23, 1662803}
// n < 3,474,749,660,383    {6 : primes <= 13}
// n < 2^64  {7 : 2, 325, 9375, 28178, 450775, 9780504, 1795265022}
bool millerRabin(ll n, ll a) {
    if (n < 2) return 0;
    if ((a = a%n) == 0) return 1;
    if (n & 1 ^ 1) return n == 2;
    
    ll tmp = (n - 1) / ((n - 1) & (1 - n));
    ll t = log2((n - 1) & (1 - n)), x = 1;
    for (; tmp; tmp >>= 1, a = a*a%n)
        if (tmp & 1) x = x * a % n;
    if (x == 1 || x == n - 1) return 1;
    while (--t)
        if ((x = x*x%n) == n - 1) return 1;
    return 0;
}
\end{lstlisting}

    \subsection{應用}
    我們可以利用以上的質數篩選法進行質因數分解,以埃氏篩法為例,
    如果我們的陣列改存其中一個可以整除他的數,那我們就可以在$O(\log{n})$
    的時間內完成質因數分解。

\begin{lstlisting}[caption=埃氏篩法]
const int N=1e6+10;
using ll=long long;
int div[N];

void init(){
    for(int &a:div) a=-1;
    div[0]=div[1]=0;
    
    for(ll i=2; i*i<=N; i++){
        if(div[i]=-1){
            div[i]=i;
            for(ll j=i*i;j<=N;j+=i){
                div[j]=i;
            }
        }
    }
}

// 這個函式會回傳一個map,裡面包含q的所有的質因數與次方。
// 不過這也導致他的時間複雜度多一個log
map<int,int> PrimeFactor(int q){
    map<int,int> ret;
    while(q>0){
        ret[div[q]]++;
        q/=div[q];
    }
    return ret;
}

// The first is the prime factor, and the second is the power.
void output(map<int,int> &mp){
    for(auto &p:mp){
        cout<<p.first<<" "<<p.second<<"\n";
    }
}
\end{lstlisting}


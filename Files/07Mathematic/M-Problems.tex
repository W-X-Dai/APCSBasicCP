\section{範例與練習}
    \subsection{數論}

    \problem CF 1553A Digits Sum

    \textbf{題目敘述}

    讓我們定義 $S(x)$ 為數字 $x$ 在十進制系統中各位數的總和。例如,$S(5)=5$,$S(10)=1$,$S(322)=7$。

    如果 $S(x+1) < S(x)$,我們將稱整數 $x$ 為有趣的。在每個測試中,
    你將會給定一個整數 $n$。你的任務是計算出在 $1 \le x \le n$ 範圍內有多少個有趣的整數 $x$。

    \textbf{輸入說明}

    第一行包含一個整數 $t$ ($1 \le t \le 1000$)- 測試案例的數量。

    接下來的 $t$ 行,第 $i$ 行包含一個整數 $n$ ($1 \le n \le 10^9$),表示第 $i$ 個測試案例的數字 $n$。

    \textbf{輸出說明}

    輸出 $t$ 個整數,第 $i$ 個整數應該是第 $i$ 個測試案例的答案。

    \textbf{範例測試}

    \begin{tabular}{|m{7cm}|m{7cm}|}
        \hline
        範例輸入 1 & 範例輸出 1 \\
        \hline
        \verb|5| & \verb|0| \\
        \verb|1| & \verb|1| \\
        \verb|9| & \verb|1| \\
        \verb|10| & \verb|3| \\
        \verb|34| & \verb|88005553| \\
        \verb|880055535| & \\
        \hline
    \end{tabular}

    \problem CF 1542C Strange Function

    \textbf{題目敘述}

    讓 $f(i)$ 表示最小的正整數 $x$ 使得 $x$ 不是 $i$ 的因數。

    計算 $\displaystyle\sum_{i=1}^n f(i)$ 模 $10^9+7$。換句話說,計算 $f(1)+f(2)+\cdots+f(n)$ 模 $10^9+7$。

    \textbf{輸入說明}

    第一行包含一個整數 $t$ ($1 \leq t \leq 10^4$),表示測試案例的數量。

    接下來的 $t$ 行為 $t$ 個測試案例,每個測試案例只包含一個整數 $n$ ($1 \leq n \leq 10^{16}$)。

    \textbf{輸出說明}

    對於每個測試案例,輸出一個整數 $ans$,其中 $ans=\displaystyle\sum_{i=1}^n f(i)$ 模 $10^9+7$。

    \textbf{範例測試}

    \begin{tabular}{|m{7cm}|m{7cm}|}
        \hline
        範例輸入 1 & 範例輸出 1 \\
        \hline
        \verb|5| & \verb|0| \\
        \verb|1| & \verb|1| \\
        \verb|9| & \verb|1| \\
        \verb|10| & \verb|3| \\
        \verb|34| & \verb|88005553| \\
        \verb|880055535| & \\
        \hline
    \end{tabular}

    \problem CF 913A Modular Exponentiation

    \textbf{題目敘述}

    以下問題是一個眾所周知的問題:給定整數 $n$ 和 $m$,計算

    $$2^n \pmod m$$

    其中 $2^n = 2 \times 2 \times \cdots \times 2$(共 $n$ 個因子),並且 $\mod$ 表示除法的餘數。

    現在請你解決這個「反向」問題。給定整數 $n$ 和 $m$,計算

    $$m \pmod {2^n}$$

    \textbf{輸入說明}

    第一行包含一個整數 $n$($1 \leq n \leq 10^8$)。

    第二行包含一個整數 $m$($1 \leq m \leq 10^8$)。

    \textbf{輸出說明}

    輸出一個整數,表示 $m \pmod {2^n}$ 的值。

    \textbf{範例測試}

    \begin{tabular}{|m{7cm}|m{7cm}|}
        \hline
        範例輸入 1 & 範例輸出 1 \\
        \hline
        \verb|4| & \verb|10| \\
        \verb|42| & \\
        \hline
        範例輸入 2 & 範例輸出 2 \\
        \hline
        \verb|1| & \verb|0| \\
        \verb|58| & \\
        \hline
    \end{tabular}

    \problem 洛谷P1082 [NOIP2012 提高組] 同餘方程

    \textbf{題目敘述}

    求關於 $x$ 的同餘方程 $a x \equiv 1 \pmod {b}$ 的最小正整數解。

    \textbf{輸入說明}

    一行,包含兩個整數 $a,b$,用一個空格隔開。

    $2 \leq a, b \leq 2,000,000,000$

    \textbf{輸出說明}

    一個整數 $x_0$,即最小正整數解。輸入數據保證一定有解。

    \textbf{範例測試}

    \begin{tabular}{|m{7cm}|m{7cm}|}
        \hline
        範例輸入 1 & 範例輸出 1 \\
        \hline
        \verb|3 10| & \verb|7| \\
        \hline
    \end{tabular}

    \problem 洛谷P1495 【模板】中國剩餘定理(CRT)/ 曹沖養豬

    \textbf{題目敘述}

    自從曹沖搞定了大象以後,曹操就開始捉摸讓兒子幹些事業,
    於是派他到中原養豬場養豬,可是曹沖滿不高興,於是在工作中馬馬虎虎,
    有一次曹操想知道母豬的數量,於是曹沖想狠狠耍曹操一把。舉個例子,
    假如有 $16$ 頭母豬,如果建了 $3$ 個豬圈,剩下 $1$ 頭豬就沒有地方安家了。
    如果建造了 $5$ 個豬圈,但是仍然有 $1$ 頭豬沒有地方去,
    然後如果建造了 $7$ 個豬圈,還有 $2$ 頭沒有地方去。
    你作為曹總的私人秘書理所當然要將准確的豬數報給曹總,你該怎麼辦?

    \textbf{輸入說明}

    第一行包含一個整數 $n$ —— 建立豬圈的次數,接下來 $n$ 行,
    每行兩個整數 $a_i, b_i$,表示建立了 $a_i$ 個豬圈,
    有 $b_i$ 頭豬沒有去處。你可以假定 $a_1 \sim a_n$ 互質。

    $1 \leq n \le 10$,$0 \leq b_i < a_i \le 100000$,$1 \leq \prod a_i \leq 10^{18}$

    \textbf{輸出說明}

    輸出包含一個正整數,即為曹沖至少養母豬的數目。

    \textbf{範例測試}

    \begin{tabular}{|m{7cm}|m{7cm}|}
        \hline
        範例輸入 1 & 範例輸出 1 \\
        \hline
        \verb|3| & \verb|16| \\
        \verb|3 1| & \\
        \verb|5 1| & \\
        \verb|7 2| & \\
        \hline
    \end{tabular}

    \problem AtCoder ABC 186E Throne

    \textbf{題目敘述}

    我們有 $N$ 把椅子排成一個圓環,其中一把是王座。

    高橋最初坐在距離王座順時針方向 $S$ 把椅子的位置上。現在,他會重複進行以下動作。

    動作:往順時針方向,走到距離他目前所坐的椅子 $K$ 把椅子的位置上。

    他將在第幾次動作後首次坐在王座上?如果他永遠不會坐在上面,請報告 $-1$。

    你需要解決 $T$ 個測試案例。

    \textbf{輸入說明}

    輸入以以下格式給出。第一行的格式如下

    $T$

    然後,接下來的 $T$ 行表示 $T$ 個測試案例。每行的格式如下:

    $N \; S \; K$

    $1 \leq T \leq 100$,$2 \leq N \leq 10^9$,
    $1 \leq S < N$,$1 \leq K \leq 10^9$

    \textbf{輸出說明}

    輸出包含一個正整數,即為曹沖至少養母豬的數目。

    \textbf{範例測試}

    \begin{tabular}{|m{7cm}|m{7cm}|}
        \hline
        範例輸入 1 & 範例輸出 1 \\
        \hline
        \verb|4| & \verb|2| \\
        \verb|10 4 3| & \verb|-1| \\
        \verb|1000 11 2| & \verb|249561088| \\
        \verb|998244353 897581057 595591169| & \verb|3571| \\
        \verb|10000 6 14| & \\
        \hline
    \end{tabular}

    \problem ZJ d308 巧克力冒險工廠

    \textbf{題目敘述}

    哲緯身為CRC的社長,現在遇上了大麻煩...

    因為資訊營的到來,他不惜成本,向世界上最大的巧克力工廠─旺卡公司訂購了塊巧克力要分給來參加資訊營的同學。

    現在哲偉利用了特殊的管道知道了總共會有m名小隊員會來參加資訊營,為了公平起見,他要把這些巧克力平分給小隊員們。

    不過現在問題來了,很可能巧克力沒有辦法被小隊員均分完、也就是有剩下,好巧不巧的是出產巧克力的工廠老闆Willy Wonka有特殊的怪癖,

    他命令哲緯不可以剩下巧克力,也不能把多出來的巧克力退還給他,不然Willy Wonka就要讓工廠裡的矮人們把哲緯丟進巧克力河裡。

    在左右為難之下...........................................

    哲緯決定要把剩下來的巧克力全部吃光光~( ̄▽ ̄)~(_△_)~( ̄▽ ̄)~(_△_)~( ̄▽ ̄)~

    你能幫幫忙算一下哲緯要負責吃掉的巧克力究竟有多少塊,讓他能夠有心理準備嗎?

    \textbf{輸入說明}

    兩個正整數n,m,代表有n塊巧克力及m個小隊員。

    我們保證:

    $n<10^1000000$,$m<1000000$

    \textbf{輸出說明}

    一個整數,代表哲緯要負責吃掉的巧克力塊數。

    \textbf{範例測試}

    \begin{tabular}{|m{7cm}|m{7cm}|}
        \hline
        範例輸入 1 & 範例輸出 1 \\
        \hline
        \verb|9999 1188| & \verb|495| \\
        \hline
    \end{tabular}

    \subsection{組合}

    \problem CF 610A Pasha and Stick

    \textbf{題目敘述}

    Pasha有一根木棍,其長度為一個正整數 $n$。他想要進行三次切割,將木棍分成四個部分。每個部分的長度必須是正整數,而這些部分的長度總和為 $n$。

    Pasha喜歡長方形,但討厭正方形,所以他想知道有多少種方式可以將木棍切割成四個部分,使得這些部分可以組成一個長方形,但無法組成一個正方形。

    你的任務是幫助Pasha計算這種切割方式的數量。如果兩種切割方式中存在一個整數 $x$,使得第一種方式中長度為 $x$ 的部分數量與第二種方式中的數量不同,
    則這兩種方式被視為不同的方式。

    \textbf{輸入說明}

    輸入的第一行包含一個正整數 $n$($1 \leq n \leq 2 \times 10^9$),代表Pasha的木棍的長度。

    \textbf{輸出說明}

    輸出一個整數,表示將Pasha的木棍切割成四個部分的方式數量,使得可以通過連接這些部分的端點形成一個長方形,但無法形成一個正方形。

    \textbf{範例測試}

    \begin{tabular}{|m{7cm}|m{7cm}|}
        \hline
        範例輸入 1 & 範例輸出 1 \\
        \hline
        \verb|6| & \verb|1| \\
        \hline
        範例輸入 2 & 範例輸出 2 \\
        \hline
        \verb|20| & \verb|4| \\
        \hline
    \end{tabular}

    \problem CF 52B Right Triangles
    
    \textbf{題目敘述}

    給定一個 $n \times m$ 的場地,場地上只包含句點('.')和星號('*')。
    你的任務是計算所有頂點位於星號('*')單元格中心的、兩邊平行於場地邊的直角三角形的數量。直角三角形是指其中一個角是直角(即90度角)的三角形。

    \textbf{輸入說明}

    第一行包含兩個正整數 $n$ 和 $m(1 \le n, m \le 1000)$。接下來的 $n$ 行,每行包含 $m$ 個字符,描述了場地的佈局。只會出現 '.' 和 '*'。

    \textbf{輸出說明}

    輸出一個整數,表示場地中的正方形三角形的總數。請不要在 C++ 中使用 \verb|%lld| 格式說明符來讀取或寫入 64 位整數。最好使用 cout(也可以使用 \verb|%I64d|)。

    \textbf{範例測試}

    \begin{tabular}{|m{7cm}|m{7cm}|}
        \hline
        範例輸入 1 & 範例輸出 1 \\
        \hline
        \verb|2 2| & \verb|1| \\
        \verb|**| &\\
        \verb|*.| & \\
        \hline
        範例輸入 2 & 範例輸出 2 \\
        \hline
        \verb|3 4| & \verb|9| \\
        \verb|*..*| &\\
        \verb|.**.| & \\
        \verb|*.**| &\\
        \hline
    \end{tabular}
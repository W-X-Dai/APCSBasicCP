\section{模運算}
    就是跟餘數有關的東西。

    \subsection{同餘}
    $a \equiv b \pmod m$,我們稱為$a$ 同餘於 $b$ 模 $m$。
    例如$9 \equiv 1 \pmod 4$,$9 \equiv -3 \pmod 4$。

    定義上,$a \equiv b \pmod m$ 等價於 $m \mid (a - b)$
    $\Leftrightarrow  a = b + mz\ (z \in \mathbb{Z})$

    \subsection{基本運算}

    $(A + B) \% m = (A \% m + B \% m) \% m$

    $(A - B) \% m = (A \% m - B \% m + m) \% m$

    $(A \times B) \% m = ((A \% m) \times (B \% m)) \% m$

    那除法呢?

    \subsection{模逆元}

    又稱為模反元素,是個有魔法的東西(誤)。

    如果 $ax \equiv 1 \pmod m$,則 $x$ 是 $a$ 模 $m$ 下的模逆元,
    記為 $x \equiv a^{-1} \pmod m$,不過要注意的是他不是分數喔。

    $(a \div b) \pmod m$ 可以視為 $\implies (a \times b^{-1}) \pmod m$。

    至於要怎麼找,我們有兩個方法。

    \textbf{費馬小定理}

    $a \in \mathbb{Z}, p$ 是質數,
    則$a^p \equiv a \pmod p$,經過基本代數運算後,可以得到
    $a^{p-2} \equiv a^{-1} \pmod p$

    證明 \url{https://hackmd.io/@Ccucumber12/Hy8nj-fxt#/3/10} 。

    實作用快速冪,可以得到$O(\log p)$複雜度解法。

    \textbf{EXT GCD}

    $ax + by = \gcd(a,b)$ 首先找出 $ax = 1 \pmod p$,
    這將導致 $ax - 1 = pz$,也就是$ax - pz = 1$
    $\implies ax + pz = gcd(a,p)$
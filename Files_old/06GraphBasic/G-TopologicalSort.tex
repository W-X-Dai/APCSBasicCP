\section{拓鋪排序}
    \subsection{概念}
    在講拓鋪排序前,我們要先講DAG(Directed Acyclic Graph),也就是有向無環圖。

    有向無環圖,顧名思義就是邊有方向,然後不存在任何的環,也就是說,
    就算你沒有加上isv判定,也不會有TLE的問題。

    Figure 6.1 就是一個DAG的範例。

    至於什麼是拓鋪排序,其實我也忘了,所以我去查資料(太少用),
    是這樣的,如果有一條邊$(a,b)$,則我們說a一定要在b之前被走訪過,

    我們可以藉由計算入度(in degree)來處理,如果入度為0就將他push進到queue裡面。

    \begin{tip}
        入度ind[x]就是有幾條邊指向x。
    \end{tip}

    \subsection{實作}

\begin{lstlisting}[caption=拓鋪排序]
vector<int> ans;
void topological_sort(){
    queue<int> topo;
    int f=0,b=0;
    for(int i=0;i<n;i++){
        if(ind[i]==0){
            topo.push(i);
            b++;
        }
    }
    
    while(!topo.empty()){
        int now=topo.front();
        topo.pop();
        ans.push_back(now);
        for(auto next:g[now]){
            if(--ind[next]==0){
                topo.push(next);
                b++;
            }
        }
    }
}
\end{lstlisting}

    \subsection{範例與練習}

    \problem TIOJ 1092 A.跳格子遊戲

    \textbf{題目敘述}

    Mimi, Moumou兩人是青梅竹馬,從小就玩在一起(雖然他們現在也才小學三年級而已)。愛玩的兩人,平常的遊戲玩久了之後就覺得無聊沒勁,常常湊在一起發明新的遊戲。

    這天他們又開始在設計新遊戲了,遊戲的規則如下:

    先在地上畫N個圓圈,編號1到N,並規定1號圓圈是起點、N號是終點

    在這N個圓圈之間任意互相畫”單向”箭頭,箭頭起點和終點各有一圓圈且兩圓圈不為同一圓,假設有一箭頭從圓a連到圓b,則遊戲中可以從a跳到b。

    檢查1, 2步驟所畫出的圖,確保任何編號1 ~ N-1的圓都有路徑可以走到終點,同時圖中也不可以有迴圈產生(也就是對於任何一個圓,絕對不會有路徑可以從自己走到自己),對任兩圓a, b最多只有一個箭頭從a指向b。

    兩個人進行遊戲,開始時先由一人站在起點(1號圓),另一個人站在圖外。

    遊戲中假設甲站在一個圓 C 中而乙在圖外,則乙要從 C 所指出去的箭頭中選一個箭頭,站到這個箭頭所指的圓上,然後甲則離開C走到圖外。

    兩人重複步驟5的動作,直到其中一人到達終點,到達終點的人為贏家。

    \textbf{輸入說明}

    輸入檔含有多筆測資,每筆資料第一行有兩個數字 $N,E(1 \le N \le 10000, \; E \le 10N)$。接下來 E 行每行有兩個數字 $a, b(1 \le a, b \le N)$,表示有箭頭從 a 指向 b。你可以假設輸入檔所代表的圖都是有效的。最後一行則是遊戲開始時站在起點上的人名。N 和 E 都是零``0 0”表示檔案結束。

    \textbf{輸出說明}

    假設遊戲為Mimi和Moumou兩人進行,且兩人皆十分聰明,如果有必勝的走法那就一定會贏。

    對每筆測資輸出遊戲結束後贏家的名字(Mimi或Moumou),佔一行。

    \textbf{範例測試}

    \begin{tabular}{|m{7cm}|m{7cm}|}
        \hline
        範例輸入 1 & 範例輸出 1 \\
        \hline
        \verb|5 6|  & \verb|Moumou| \\
        \verb|1 2|  &\\
        \verb|1 3|  &\\
        \verb|2 3|  &\\
        \verb|2 4|  &\\
        \verb|3 5|  &\\
        \verb|4 5|  &\\
        \verb|Mimi|  &\\
        \verb|0 0|  &\\
        \hline
    \end{tabular}

    \problem Sprout OJ 165 陣線推進

    \textbf{題目敘述}

    如果你有玩過策略遊戲,尤其是 AOC,的話,一定曉得「戰線」是一種很重要的概念。
    大致上這概念可以這樣說明:如果對方設置陣地得宜,那麼在攻破某些陣地之前,
    一定要先攻破某些陣地,否則會導致我軍不是被城牆擋住,
    就是會被兩邊夾擊、全軍覆沒。

    一般來說兩個玩家單挑對決時,因為通常都是互相快攻,彼此的陣地都還只有本陣,
    這方面問題就不大;但是當玩家數量變多,例如變成 4v4 團體戰的時候,
    隨著遊戲時間增常,陣地之間的關係也容易變得非常複雜,
    想攻破 A 要攻破 B 和 C、想攻破 B又要先攻破D 和E 等等。
    為了避免兵力的大幅損傷,你派出了大量的斥候犧牲小我完成大我,
    先確認了各個陣地的配置以及位置,接著便是擬定作戰計畫的時候了。
    如果已經知道要攻破某個陣地前要先攻破哪些陣地,你能否給出一個方案
    使得根據這個方案進行陣地攻略可以滿足所有要求條件、
    在全程都不腹背受敵的情況下與對方決一死戰呢?

    \textbf{輸入說明}

    輸入的第一行是一個正整數 T。接下來會有 
    T 組測試資料,每組測試資料的第一行是兩個非負整數 
    n,m,分別代表陣地的數量與陣地間的依賴關係數量;第二行到第 
    $m+1$ 行每行由兩個整數 a 和 b $(0\le a,b \le n-1)$組成,代表要攻破陣地 b 之前必須先攻破陣地 a,
    其中陣地從 $0$ ~ $n-1$依序編號。

    \textbf{輸出說明}

    如果存在題目所要求的方案,為了方便起見請輸出字典序最小的一個。

    否則代表一場喋血的戰役在所難免,只好輸出"QAQ"(不含引號)了。

    \begin{tip}
        誰說只能用queue呢?
    \end{tip}

    
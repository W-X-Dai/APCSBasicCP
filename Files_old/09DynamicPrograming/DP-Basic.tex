\section{基本知識}
    \subsection{什麼是動態規劃}
    其實動態規劃並沒有動態,也不是規劃,那他會什麼會有這樣的名稱?
    答案是因為,想出這樣的解體方法的人,他的上司討厭數學,所以他就取了一個,
    與數學毫無關係的名字。

    實際上,動態規劃主要用以解決兩種問題,極值問題與計數問題。
    例如:數學那一章我們就用到他來算組合數。

    使用動態規劃前,我們需要先分析題目,如果題目具有\textbf{最優子結構}才能使用。

    \subsection{最優子結構}
    你可能想說,最優子結構到底是三$\cdots$ ,其實就是,
    局部的最佳解可以帶領我們找到整體的最佳解,這樣的結構。

    另外就是,如果是計數問題,我們的局部解可以引導我們計算
    出全域的解。

    \subsection{遞迴式表達}
    以後我們可能會表達許多遞迴式,以數學來說會用這樣的方式表達。

    $$
    \begin{cases}
        dp_1=dp_2=1 \\ 
        dp_n=dp_{n-1}+dp_{n-2}, n > 2
    \end{cases}
    $$

    \subsection{遞迴紀錄法}
    一個最方便又偷懶的方式就是用遞迴直接執行,但重複的運算透過記錄,
    使其不再重複。如果以費氏數列為例。

\begin{lstlisting}[caption=費氏數列的遞迴DP]
using ll=long long;
ll dp[N];
int fib(int n){
    if(n==1 || n==2){
        return 1;
    }

    if(dp[n]!=0){
        return dp[n];
    }

    return dp[n]=fib(n-1)+fib(n-2);
}
\end{lstlisting}

    \subsection{迴圈推進法}
    如果,我們可以保證,所有我們在計算$dp_n$所需要的值都在計算他
    之前被計算完畢,則我們可以使用迴圈在陣列上面推進。同樣以費氏數列為例。

\begin{lstlisting}[caption=費氏數列的遞迴DP]
using ll=long long;
ll dp[N];
int init(int n){
    dp[1]=dp[2]=1;
    for(int i=3;i<N;++i){
        dp[i]=dp[i-1]+dp[i-2];
    }
}

int fib(int n){
    return dp[n];
}
\end{lstlisting}

    這種方式的好處是可以節省常數,也很直觀(一般而言)。

    \subsection{常規方法}
    對於那種已經有遞迴式的$dp$,其實是最不需要動腦的一種。
    而我們要處理的問題絕對不是只有這麼簡單。因此我需要告訴你們
    一些常規的解題方法。

    \begin{enumerate}
        \item 定義狀態,也就是定義$dp_i$所代表的意義。
        \item 列出轉移式,也就是定義$dp_i$與其他向之間的關係。
        \item 決定計算順序:方便我們使用``迴圈推進法''。(包含定義初始狀態)
    \end{enumerate}

    基本上只要會常規方法就可以解決所有問題(誰說的),但是常規方法
    裡面有很多原則性的東西,我們需要藉由練習,提升我們的使用能力。

    \subsection{補充}
    在常規方法裡面我們提到$dp_i$,其實,有時候我們不只有一個變數,
    所以接下來你看到$dp_{ij}$的時候不用慌張,代表他的狀態由兩個變數。

    \subsection{範例與練習}

    \problem 遞迴式實作練習,請實作出可以計算的程式。

    $$
    \begin{cases}
        dp_1=dp_2=1 \\
        dp_3=10 \\
        dp_n=3dp_{n-1}+dp_{n-2}+dp_{n-3}
    \end{cases}
    $$

    \problem 遞迴式實作練習,請實作出可以計算的程式。其中,
    $a$是另一個輸入。

    $$
    \begin{cases}
        dp_1=a_1 \\
        dp_2=\max (a_1,a_2) \\
        dp_n=\max (dp_{n-1},dp_{n-2}+a_n)
    \end{cases}
    $$

    \problem 遞迴式實作練習,請實作出可以計算的程式。其中,
    $a$是另一個輸入。

    $$
    \begin{cases}
        dp_1=a_1 \\
        dp_2=\max (a_1,2a_2) \\
        dp_n=\max (dp_{n-1}+a_n,dp_{n-2}+2a_n)
    \end{cases}
    $$
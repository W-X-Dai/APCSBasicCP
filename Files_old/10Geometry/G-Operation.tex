\section{操作}
    \subsection{面積計算}
    我們可以使用行列式計算$v$與$u$所張成的平行四邊形面積,不過值得注意的是,
    面積有方向。如果v順時針轉比較靠近u,那就是正的,否則就是負的。

    $$area(v,u)=(v_1 u_2 - v_2 u_1)$$

    \begin{proof}
        \begin{align*}
            \cos \theta &= \frac{\sum\limits_{i=1}^{n} (2\times v_i \times u_i)}{2\times |v| \times |u|} \\
            &=\frac{v_1 u_1 + v_2 u_2}{|v| \times |u|} \\
            \sin \theta &= \sqrt{1- \cos^2 \theta} \\
            &= \frac{\sqrt{(|v||u|)^2-(v_1 u_1 + v_2 u_2)}}{|v||u|}\\
            &= \frac{\sqrt{(v_1^2+v_2^2)(u_1^2+u_2^2)-(v_1 u_1 + v_2 u_2)^2}}{|v||u|}\\
            &= \frac{v_1 u_2 - v_2 u_1}{|v||u|}\\
            area(v,u) &=|v||u|\sin \theta \\
            &=v_1 u_2 - v_2 u_1
        \end{align*}
    \end{proof}

    這樣的特性可以讓我們稍稍判斷兩個向量的相對方位。

    實作上,我們會用外積(把z設為0)稱呼他。

\begin{lstlisting}[caption= 面積計算]
T cross(const Pt a,const Pt b) { return a.x*b.y - a.y*b.x;}

int ori(const Pt a,const Pt b) {// positive negative zero
    T ret=cross(a,b);
    if(ret==0) return 0;
    return (ret<0)?-1:1;
}
\end{lstlisting}

    \subsection{判斷點是否在直線上}
    我們可以藉由向量的運算得知。點$C$在直線$\overleftrightarrow{AB}$上 
    $\Leftrightarrow \overrightarrow{AC}= k \overrightarrow{AB}, \; k \in \R$。

    判斷$\overrightarrow{AC}= k \overrightarrow{AB}$是否成立的方法很簡單,因為兩個
    向量平行,所以所圍出的面積為$0$。如此以來,
    我們就可以用$cross(C-A,B-A)=0$與否判斷。

    \subsection{判斷點是否在線段上}
    我們可以藉由向量的運算得知。點$C$在線段$\overline{AB}$上,
    首先他必須在直線$\overleftrightarrow{AB}$上。接著,
    $\overrightarrow{CA}$與$\overrightarrow{CB}$的夾角要是$180^{\circ}$。

    $$\implies \overrightarrow{CA} \cdot \overrightarrow{CB} < 0$$

    \subsection{判斷線段是否相交}
    首先考慮這個情況,線段的端點在對方的線段中,那就算是相交了。

    除此之外,那就有點困難了。
    你可以解方程式之後判斷交點是否在兩個線段的範圍中。
    不過這裡我們介紹另一個方法,假設兩線段分別是AB,CD。

    \begin{align*}
        \text{cross}(\overrightarrow{AB}, \overrightarrow{AC}) \times \text{cross}(\overrightarrow{AB}, \overrightarrow{AD}) < 0 \\
        \text{cross}(\overrightarrow{CD}, \overrightarrow{CA}) \times \text{cross}(\overrightarrow{CD}, \overrightarrow{CB}) < 0
    \end{align*}

    考慮cross正負發生的時機就可以知道其正確性。

    \subsection{多邊形面積}
    考慮使用測量師公式。設多邊形頂點為$P_1,P_2, \cdots, P_n$

    $$area(P)=\sum_{i=1}^{n}\overrightarrow{OP_i}  \times \overrightarrow{OP_{i+1}}, \; 定義 \; n+1=1$$

    一個迴圈就搞定了。但是需要注意頂點必須要順時針或逆時針排序,所以以下來介紹怎麼做。

    \subsection{極角排序}
    就是讓頂點順時針或逆時針排序,所以以下來介紹怎麼做。

    首先,三角函數的
    常數極大,因為用泰勒展開式等實作,所以以下方式雖然可以使用,但是很慢。

\begin{lstlisting}[caption=極角排序 with 反三角函數]
bool cmp(Pt a,Pt b){
    return atan2(a.x,a,y)<atan2(b.x,b.y);
}
\end{lstlisting}

    另一個方式是可以使用外積與座標平面特性,因為外積的適用範圍只有$180^{\circ}$,
    所以我們必須拆開上下(或左右)半平面。

\begin{lstlisting}[caption=極角排序 with cross]
bool cmp(Pt a,Pt b){
    bool f1=a.x<0 || (a.x==0 && a.y<0);
    bool f2=b.x<0 || (b.x==0 && b.y<0);
    if(f1!=f2) return f1<f2;
    return cross(a,b)>0;
}
\end{lstlisting}

    \subsection{凸包}
    給你一堆點,問你可以包住這些所有點的最小凸多邊形。

    這樣的多邊形一定可以藉由對x,y做字典順序,並嘗試
    一個一個加入點。

    如果角度超過$180^{\circ}$,就把裡面的點拿出來。

    這樣的操作做兩次,一次從$1 \to n$,另一次從$n \to 1$。

    判斷角度超過$180^{\circ}$,若 
    $\text{cross}(\overrightarrow{AB}, \overrightarrow{BC}) \leq 0$ 
    則 $B$ 不在凸包上。

\begin{lstlisting}[caption=凸包]
vector<Pt> ConvexHull(vector<Pt> ds){
    sort(ds.begin(),ds.end(),CmpByXY);

    vector<Pt> ret;
    int k=0;
    for(auto i:ds){
        while(k>=2 && ori(ret[k-1]-ret[k-2],i-ret[k-1])<=0) {
            ret.pop_back();
            k--;
        }

        ret.emplace_back(i);
        k++;
    }

    ret.pop_back();
    k--;

    reverse(ds.begin(),ds.end());

    int hf=k;
    for(auto i:ds){
        while(k-hf>=2 && ori(ret[k-1]-ret[k-2],i-ret[k-1])<=0) {
            ret.pop_back();
            k--;
        }

        ret.emplace_back(i);
        k++;
    }
    ret.pop_back();// not count right element twice
    return ret;
}
\end{lstlisting}

    凸包可以應用在求最遠點對,因為如果值域的範圍是$C$,
    那凸包上最多會有$\sqrt{C}$個點。枚舉所有點對的複雜度為
    $O(C)$。

    另外DP優化也有可能會用到。

    \subsection{其他資源}
    
    \url{https://hackmd.io/@Ccucumber12/BJeOhtzbF#/}

    \includegraphics[width=0.2\textwidth]{../Images/Vector1.png}

    
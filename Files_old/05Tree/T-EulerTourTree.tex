\section{樹上尤拉路徑}
    \subsection{什麼是尤拉路徑}
    通常而言,尤拉路徑是指所有的邊都只通過一次的路徑,
    不過樹上很明顯沒有這東西。所以樹上的尤拉路徑指的是每個邊
    都通過兩次的路徑。我們可以藉由DFS來記錄這樣的路徑。

    \subsection{用途}
    藉由這方法,配合資料結構,同樣可以$O(\log(n))$完成LCA的查詢。不僅如此,
    有一些其他問題也會用到這個操作。

    \subsection{建構}
\begin{lstlisting}[caption=樹上尤拉路徑]
vector<int> g[100010] ;
vector<int> et ;
int in[100010], out[100010] ;
 
void dfs(int x, int p) {
    et.push_back(x) ;
    in[x] = et.size() - 1 ;
    for(auto i:g[x]) if(i != p) {
        dfs(i, x) ;
        et.push_back(x) ;
    }
    out[x] = et.size() - 1 ;
}
\end{lstlisting}

    如果你希望它可以用來解決LCA問題,你還需要多存一個變數,
    就是深度。而此時兩個節點的LCA就會是兩個節點之間的深度最小值位置。

    \subsection{範例與練習}
    \problem CF 620 E New Year Tree

    \textbf{題目敘述}

    新年假期結束了,但 Resha 不想丟掉新年樹。
    他邀請了他最好的朋友 Kerim 和 Gural 幫助他重新裝飾新年樹。

    新年樹是一個有 n 個頂點且根在頂點 1 的無向樹。

    你需要處理兩種類型的查詢:

    \begin{enumerate}
        \item 將頂點 v 的子樹中的所有頂點顏色更改為顏色 c。
        \item 查找頂點 v 的子樹中不同顏色的數量。
    \end{enumerate}

    \textbf{輸入說明}

    第一行包含兩個整數 $n$ 和 $m$ $(1 \le n, m \le 4\times 10^5)$ — 樹中頂點的數量和查詢的數量。

    第二行包含 $n$ 個整數 $c_i$ $(1 \le c_i \le 60)$ — 第 $i$ 個頂點的顏色。

    接下來的 $n-1$ 行,每行包含兩個整數 $x_j$ 和 $y_j$ $(1 \le x_j, y_j \le n)$ — 第 $j$ 條邊的兩個頂點。保證給出的是正確的無向樹。

    最後的 $m$ 行包含查詢的描述。
    每個描述以整數 $t_k (1 \le t_k \le 2)$ 開始 — 第 $k$ 個查詢的類型。
    對於第一類型的查詢,接下來是兩個整數 $v_k$ 和 $c_k$ $(1 \le v_k \le n, 1 \le c_k \le 60)$ — 
    子樹將用顏色 $c_k$ 重新著色的頂點號碼 $v_k$。對於第二類型的查詢,接下來是一個整數 $v_k$ $(1 \le v_k \le n)$ — 
    需要查找其子樹中不同顏色數量的頂點號碼。
    
    \textbf{輸出說明}

    對於每個第二類型的查詢,輸出一個整數 $a$ — 查詢中給定的頂點的子樹中不同顏色的數量。

    每個數字應該以單獨的行按照輸入中出現的查詢順序進行輸出。

\section{樹重心}
    \author{Shang Jhe Li}

    \subsection{概念}
    樹的重心就是拔除他後,形成的若干棵樹分別的節點數量中的最大值,會最小的那個節點。

    也可以說,以重心為根後,所有子樹的大小都不超過總結點的一半。

    如果我們想要知道樹重心在哪裡的話,我們可以應用DP技巧。
    首先,我們同樣是隨意找一個節點當作根,接著向下DFS。
    接著,對於每個節點,我們都找出它的最大子樹。
    最後,不要忘記,如果把它當作根節點,那它上面的所有節點都將會是他的子樹,
    所以還要考慮上面的所有節點。

    \subsection{實作}

\begin{lstlisting}[caption=樹重心]
#include<bits/stdc++.h>
using namespace std;
#define INF 1000000000

vector<int> child[100010];
// sz[x] 是 x 的子樹大小,dp[x] 是 x 下面最大的子樹大小。
int sz[100010],dp[100010],ans=-1,n,mn=INF;
// vt 就是 isv,表示有沒有經過這個點。
bool vt[100010];

void dfs(int a){
    sz[a]=1;
    for(auto c:child[a]){
        if(!vt[c]){
            vt[c]=true;
            dfs(c);
            sz[a]+=sz[c];
            dp[a]=max(dp[a],sz[c]);
        }
    }
    dp[a]=max(dp[a],n-sz[a]);
    if(mn>dp[a]){
        mn=dp[a];
        ans=a;
    }else if(mn==dp[a] && ans>a){
        ans=a;
    }
}

int main(){
    ios::sync_with_stdio(0);cin.tie(0);
    
    int t;
    cin>>t;
    for(int iptNum=0;iptNum<t;iptNum++){
        mn=INF;
        for(int i=0;i<100010;i++){
            child[i].clear();
            sz[i]=0;
            dp[i]=0;
            vt[i]=false;
        } 
        cin>>n;
        for(int i=1;i<n;i++){
            int c,p;
            cin>>p>>c;
            child[p].emplace_back(c);
            child[c].emplace_back(p);
        }
        dfs(0);
        cout<<ans<<"\n";
    }
    return 0;
}
\end{lstlisting}

    \subsection{範例與練習}

    \problem CF 708C Centroids

    \textbf{題目敘述}

    給定一個由$n$個節點組成的樹。如果從樹中刪除該節點後,
    每個連通分量的大小都不超過$\frac{n}{2}$,則該節點稱為重心。

    現在,你可以進行最多一次的邊替換操作。邊替換是指從樹中刪除一條邊(保留相應的節點),
    然後插入一條新的邊(不添加新的節點),使得圖形仍然是一棵樹。
    你需要判斷每個節點是否可以透過最多一次的邊替換成為重心。

    \textbf{輸入說明}

    第一行包含一個整數$n(2 \le n \le 4 \times 10^5)$,表示樹中的節點數量。
    接下來的$n-1$行,每行包含一對節點索引$u_i$和$v_i(1 \le u_i, v_i \le n)$,表示相應邊的兩個端點。
    
    \textbf{輸出說明}

    輸出n個整數,第i個整數等於1表示第i個節點可以透過最多一次的邊替換成為中心節點,等於0則表示不能。

    \textbf{範例測試}

    \begin{tabular}{|m{7cm}|m{7cm}|}
        \hline
        範例輸入 1 & 範例輸出 1 \\
        \hline
        \verb|5|  & \verb|1 0 0 0 0| \\
        \verb|1 2| & \\
        \verb|1 3|  & \\
        \verb|1 4|  & \\
        \verb|1 5|  & \\
        \hline
        範例輸入 2 & 範例輸出 2 \\
        \hline
        \verb|3|  & \verb|1 1 1| \\
        \verb|1 2| & \\
        \verb|2 3|  & \\
        \hline
    \end{tabular}

    \begin{tip}
        先找出重心,再利用它的性質。
    \end{tip}

    \problem 洛谷P1395 會議

    \textbf{題目敘述}

    在一個村莊裡住著$n$個村民,這$n$個村民的家通過$n-1$條路徑相連,
    每條路徑的長度為1。現在,村長計劃在其中一個村民的家裡舉辦一場會議,
    村長希望選擇一個村民的家作為會議地點,使得所有村民到會議地點的距離之和最小。
    如果有多個村民的家都滿足條件,則選擇村民編號最小的家作為會議地點。

    \textbf{輸入說明}

    第一行包含一個整數$n$,表示村民的數量。

    接下來的$n-1$行,每行包含兩個整數$a$和$b$,
    表示村民$a$的家和村民$b$的家之間存在一條路徑。

    $n \le 5 \times 10^4$
    
    \textbf{輸出說明}

    輸出一行,包含兩個整數$x$和$y$。
    $x$表示會議地點所在的村民家的編號。
    $y$表示所有村民到會議地點的距離之和的最小值。

    \textbf{範例測試}

    \begin{tabular}{|m{7cm}|m{7cm}|}
        \hline
        範例輸入 1 & 範例輸出 1 \\
        \hline
        \verb|4|  & \verb|2 4| \\
        \verb|1 2| & \\
        \verb|2 3|  & \\
        \verb|3 4|  & \\
        \hline
    \end{tabular}

    
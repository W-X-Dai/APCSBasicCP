\section{滑動窗口}
    \subsection{概念}
    有時候我們會遇到這樣的問題,問題希望你找出所有長度為$k$的連續區間的最大最小值。
    這時候,我們可以使用滑動窗口的技巧完成這樣的問題。
    
    滑動窗口主要關注元素的進出,以最大值為例,我們可以維護一個雙向佇列(deque)。
    這個deque裡面存放由大到小的元素位址。

    以下表為例,假設$k=4$。

    \begin{center}
    \begin{tabular}[ht]{c|c|c|c|c|c|c}
        陣列位址 & 1 & 2 & 3 & 4 & 5 & 6 \\
        \hline
        陣列元素 & 5 & 2 & 4 & -3 & -1 & 9
    \end{tabular}
    \end{center}

    則區間$1-4$的最大值為$5$。
    區間$2-5$則為$4$,依此類推。

    我們可以藉由對deque的操作,處理每一個區間的最大值。

\begin{lstlisting}[caption=維護最大值deque]
const int N=100010;
int a[N];
deque<int> dq;

void op(int n,int k){
    while(dq.front()<n-k){
        dq.pop_front();
    }

    while(a[dq.back()]<a[n]){
        dq.pop_back();
    }

    dq.push_back(n);
}
\end{lstlisting}

    front會存最大值的位址。只要重複這樣的步驟就可以完成對所有區間的
    最大值查詢。我就不列出完整的演算法了。

    \subsection{複雜度}
    看到完整演算法的雙層迴圈我們可能會以為他的複雜度是$O(n^2)$,
    但是其實不然,因為我們的元素最多就是進去還有出來deque一次。
    所以平均的時間複雜度是$O(n)$。

    通常滑動窗口的技術都可以將複雜度下降,因為我們關注進出,這樣的想法在進階技術會再出現一次,也就是莫隊算法。
    這一題我們也可以使用priority\verb|_|queue,不過時間複雜度會變為$O(n\log{(n)})$。

    \subsection{範例與練習}

    \problem 最長不重複區間

    \textbf{題目敘述}

    給你一個長度為$n$的陣列,請求出最長的連續區間,
    使區間內任兩個數字都不相同。

    \textbf{輸入說明}

    第一行輸入一個數字$n$。

    第二行輸入$n$個數字,表示陣列本身。

    \textbf{輸出說明}

    輸出一個數字,表示滿足這樣的條件的區間長度。

    \textbf{範例測試}

    \textbf{輸出說明}

    輸出 $t$ 個數字 $S_n$ (記得對 $M$ 取餘),數字間要換行。

    \textbf{範例測試}

    \begin{tabular}{|m{7cm}|m{7cm}|}
        \hline
        範例輸入 1 & 範例輸出 1 \\
        \hline
        \verb|5| &  \verb|3| \\
        \verb|1 2 3 2 1| & \\
        \hline
        範例輸入 2 & 範例輸出 2 \\
        \hline
        \verb|7| &  \verb|5| \\
        \verb|1 5 3 4 2 5 2 1| & \\
        \hline
    \end{tabular}

    \problem 最小覆蓋子字串

    \textbf{題目敘述}

    给定一个字串$S$和一个字符串$T$,请在$S$中找出包含$T$所有字母的最小子字串。

    \textbf{輸入說明}

    第一行輸入一個字串$S$。

    第二行輸入一個字串$T$。

    \textbf{輸出說明}

    請輸出最小所需長度。

    \textbf{範例測試}

    \begin{tabular}{|m{7cm}|m{7cm}|}
        \hline
        範例輸入 1 & 範例輸出 1 \\
        \hline
        \verb|ADOBECODEBANC| &  \verb|4| \\
        \verb|ABC| & \\
        \hline
    \end{tabular}

    \begin{tip}
        建議同學看完資料結構後可以再回來翻這個單元,或許會有所收穫。
    \end{tip}
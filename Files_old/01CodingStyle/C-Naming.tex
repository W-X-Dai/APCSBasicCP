\section{變數命名}
    這個章節我們要來探討變數名稱的意義。在業界,變數名稱要盡可能包含所有資訊,讓共事的夥伴可以藉由變數名稱讀懂程式碼。然而,有些人認為競賽只要用盡可能簡潔的變數名稱就好。

    \subsection{在簡短與明確中抉擇}
    我承認競賽上不太可能用向實務一樣長的變數名稱,因為競賽必須與時間賽跑。但太過簡短,甚至沒有意義的名稱可能導致除錯困難,因此會有一些折衷的方案,希望盡可能同時解決兩邊的問題。

    \subsection{縮寫}
    在命名變數時,使用英文縮寫讓程式碼變的簡短,是一個常見的方式。以下是一個簡單的表格,列舉出我可能會用到的縮寫代表的意義。
    \begin{center}
        \begin{tabular}{c|c|c|c|c|c|c}
             \textbf{縮寫} & ct  & isv & mx & mn & idx & num \\
             \hline
             \textbf{意義} & count & is valid & max & min & index & number
        \end{tabular}
    \end{center}
    

    \subsection{組合型單字}
    對於組合型單字,我們可以使用兩種方式命名。

    \begin{enumerate}
        \item 加上底線,Ex: \verb|item_number|。
        \item 除了第一個單詞,後面的單詞第一個字母大寫,Ex: \verb|itemNumber|
    \end{enumerate}

    \subsection{其他}
    其他的命名方式可以參考 Code Complete 第11章「變數名稱的力量」。
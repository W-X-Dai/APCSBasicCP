%Preamble by Saintan

%%%引入Package%%%
%欲查詢各package詳細使用方法,參見https://ctan.org/pkg/<package名稱>
%如https://ctan.org/pkg/ulem,並閱讀Package documentation
    %% font & format %%
\usepackage[margin=3cm]{geometry} % 上下左右距離邊緣3cm
\usepackage{fontspec}   % 加這個就可以設定字體
\usepackage{type1cm}	% 設定fontsize用
\usepackage{titlesec}   % 設定section等的字體
\usepackage{fancyhdr}   % 頁首頁尾
\usepackage{multicol}   % 多欄 \multicols
\usepackage[normalem]{ulem}  % 字加裝飾(underline + emphasis or strikeout)
\usepackage[dvipsnames,table,xcdraw]{xcolor}     % 更廣泛的顏色配置
    %% Math %%
% \usepackage{amsmath}
\usepackage{amsthm,amssymb} % 引入 AMS 數學環境
\usepackage{mathtools}  % 新一代數學工具,修正一點amsmath的bug
\usepackage{yhmath}     % BIG math symbol,如matrix用的超大圓括弧、角括弧
\usepackage{tikz-cd}    % 含tikz,可以多畫diagram類的東西
\usepackage{thmtools}   % 定理套件
    %% graphics %%
\usepackage{graphicx}   % 旋轉、移動等圖形效果
\usepackage{wrapfig}    % 文繞圖
    %% Enhancement %%
\usepackage{titling}    % 加強 title 功能
\usepackage{tabularx}   % 加強版 table
\usepackage[shortlabels, inline]{enumitem}  % 加強版enumerate
%\usepackage[showzone=true, showseconds=false, showisoZ=false,showzoneminutes=false,calc]{datetime2} %加強版時間格式
\usepackage[square, comma, numbers, super, sort&compress]{natbib}  % cite加強版
    %% Logos & symbols %%
%\usepackage{textcomp}  % 可以打出許多好用的符號,Overleaf已預載
\usepackage{academicons}% 可以打出arXiv, Academia, Overleaf等Logo
\usepackage{wasysym}    % 可以打出各種怪符號如 \sun, \smiley, \eighthnote
\usepackage{manfnt}     %Knuth,and this is some easter egg
\usepackage{dsfont}     % 另一種雙標線字體 \mathds,多了\mathds{1 h k}
\usepackage{metalogo}   % 打出 XeLaTeX
\usepackage[scr]{rsfso} % 草寫字體

\usepackage[CJKmath=true,AutoFakeBold=3,AutoFakeSlant=.2]{xeCJK}% XeLaTeX 中日韓
%\usepackage{CJKulem} %中日韓裝飾,不可與 xeCJK 一同使用
\setCJKmainfont{AR PL KaitiM Big5}
\setCJKsansfont{AR PL KaitiM Big5}
\setCJKmonofont{AR PL KaitiM Big5}
%\setCJKmainfont[AutoFakeBold=3,AutoFakeSlant=.4]{標楷體}
\defaultCJKfontfeatures{AutoFakeBold=3,AutoFakeSlant=.4}
\newCJKfontfamily\Kai{標楷體}
\newCJKfontfamily\Hei{微軟正黑體}
\newCJKfontfamily\NewMing{新細明體}
%AR PL KaitiM Big5
\XeTeXlinebreaklocale "zh"
    %% 

    %% Code %%
\usepackage{listings}   % 用於顯示code
% \usepackage{pxfonts}
% \usepackage{showexpl}   % 用於顯示LaTeX code

    %% Other package %%
\usepackage{pgf}    % For random


% \RequirePackage[hyphens]{url}
% \usepackage{xurl} % Trying new method

\usepackage[unicode=true, pdfborder={0 0 0}, bookmarksdepth=-1]{hyperref} % ref 加強版,請儘量把 hyperref 放在最後一個引入的package
\usepackage[noabbrev, capitalize, nameinlink]{cleveref} % cleveref must be loaded after hyperref!
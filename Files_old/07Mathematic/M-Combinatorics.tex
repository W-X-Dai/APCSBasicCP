\section{組合}
    \subsection{概念}
    就是排列組合的組合,所以我們有學過這些公式。不過我們先介紹符號,
    雖然說現在應該是用不到。

    \begin{align}
        \binom{n}{k}=C^{n}_{k}= \frac{n!}{k!(n-k)!} \\
        \binom{n}{k}=\binom{n-1}{k}+\binom{n-1}{k-1}
    \end{align}

    因為這兩條公式,所以我們有兩種方式可以計算$\binom{n}{k}$。

\begin{lstlisting}[caption=計算$\binom{n}{k}$]
using ll=long long;
const ll MOD=1e9+7;

ll POW(ll a,ll x){
    ll ret=1;
    while(x>0){
        if(x&1) ret=(ret*a)%MOD;
        a=(a*a)%MOD;
        x>>=1;
    }
    return ret;
}

ll stair(int n){
    ll ret=1;
    for(int i=2;i<=n;++i){
        ret=(ret*i)%MOD;
    }
    return ret;
}

// 時間複雜度約為 O(n log MOD)
// 不過可以藉由建a!與他的反元素的表提升為O(1)
ll C(int n,int k){
    ll ret=stair(n);
    ret=(ret*POW(stair(k),MOD-2));
    ret=(ret*POW(stair(n-k),MOD-2));
    return ret;
}

// 時間複雜度 O(NK) ,但會生成一個表可以查詢任何的dp[n][k]。
const int N=1005,K=1010;
ll dp[N][K];
void C2(){
    for(int i=1;i<N;++i){
        dp[i][0]=1;
    }

    for(int i=1;i<N;++i){
        for(int j=1;j<=i;++j){
            dp[i][j]=dp[i-1][j]+dp[i-1][j-1];
        }
    }
}
\end{lstlisting}

